%=+=+=+=+=+=+=+=+=+=+=+=+=+=+=+=+=+=+=+=+=+=+=+=+=+=+=+=+=+=+=+=+=+=+=+=+=+=+=+=
%             _             _
%            | |           | |
%  _ __ ___  | |__    ___  | | ___   ___   _ __       ___   ___   _ __ ___
% | '_ ` _ \ | '_ \  / _ \ | |/ __| / _ \ | '_ \     / __| / _ \ | '_ ` _ \
% | | | | | || | | || (_) || |\__ \| (_) || | | | _ | (__ | (_) || | | | | |
% |_| |_| |_||_| |_| \___/ |_||___/ \___/ |_| |_|(_) \___| \___/ |_| |_| |_|
%
% Author: Mark H. Olson
% Website: https://mholson.com
% Github: https://github.com/mholson
%
% Created: 2021-06-09
%
% This template is designed for International Baccalaureate students who
% are looking to write an extended essay or an internal assessment using LaTeX.
% The template is based on KOMA Script scrbook class and further modified using
% the custom (mhotext.sty) style file, which was inspired by the code of Napkin
% http://web.evanchen.cc/napkin.html written by Evan Chan.
%
% It is important to know that this template can only be compiled under XeLaTeX
% or LuaLaTeX as it is making use of OTF Libertinus fonts
% via the (libertinus-otf) package being called by the mhotext.sty file.
%
% While I have done my best to ensure this template meets the requirements as
% stipulated by the International Baccalaureate, it is the responsibility
% of the student to check the official International Baccalaureate documentation
% to ensure their work meets all the specifications and requirements of any work
% submitted.  The student should also have the style of their work checked by
% their supervising teacher before submitting their final draft to the IB for
% grading.
%=+=+=+=+=+=+=+=+=+=+=+=+=+=+=+=+=+=+=+=+=+=+=+=+=+=+=+=+=+=+=+=+=+=+=+=+=+=+=+=

%=-=-=-=-=-=-=-=-=-=-=-=-=-=-=-=-=-=-=-=-=-=-=-=-=-=-=-=-=-=-=-=-=-=-=-=-=-=-=-=
% PREAMBLE :: main.tex
%=-=-=-=-=-=-=-=-=-=-=-=-=-=-=-=-=-=-=-=-=-=-=-=-=-=-=-=-=-=-=-=-=-=-=-=-=-=-=-=
%
% > > > Templates for text, math and figure elements are can be found at the end
%       of this document ... below the \end{document} line.
%=-=-=-=-=-=-=-=-=-=-=-=-=-=-=-=-=-=-=-=-=-=-=-=-=-=-=-=-=-=-=-=-=-=-=-=-=-=-=-=

\documentclass[fontsize=12pt,twoside=semi,openright,numbers=noenddot,parskip=half]{scrbook}

% > > > srchack is necessary for some packages to play nicely with KomaScript scrbook
\usepackage{scrhack}

% > > > scrbook customizations - this is where the magic happens
\usepackage{mhotext}

% > > > If you are going to be referencing multiple images from the same path
%       then it might be a good idea to set the path of where the images are
%       located, so that you do not need to manually include it each time.

\graphicspath{{\string~/GitHub/IBTeX/assets/}}

% > > > Uncomment to view labels in right margin (might need to deprecate later)
%\usepackage[inner, right]{showlabels}

% > > > Uncomment to show directory structures as shown in the documentation
\usepackage{dirtree}
% > > > The IB suggests that internal assessment be written with double line
%       spacing.  In order to enable an alternative line spacing, you
%       can make use of the setspace package.  The line after \begin{document}
%       is where you can set your spacing to meet the needs of the submission
%       requirements.
%\usepackage{setspace}

%=-=-=-=-=-=-=-=-=-=-=-=-=-=-=-=-=-=-=-=-=-=-=-=-=-=-=-=-=-=-=-=-=-=-=-=-=-=-=-=
% SETUP THE TITLE PAGE
%=-=-=-=-=-=-=-=-=-=-=-=-=-=-=-=-=-=-=-=-=-=-=-=-=-=-=-=-=-=-=-=-=-=-=-=-=-=-=-=
% > > > Enter the title of the essay
\title{IB\TeX}

% > > > Enter the the research question here
\subtitle{A Template for Writing the IB Extended Essay and Internal Assessments.}

% > > > Enter your name on the draft version ONLY.  When compiling your final
%       draft, then change the argument of author to your subject area.
%\author{Mark Olson}

\author{mholson}
% > > > It was just easier to replace the date with the word count. This is
%       not necessary for the internal assessment, but is required for the
%       extended essay.
\date{Words: 1234}
\addbibresource{references.bib}
%=-=-=-=-=-=-=-=-=-=-=-=-=-=-=-=-=-=-=-=-=-=-=-=-=-=-=-=-=-=-=-=-=-=-=-=-=-=-=-=
% DOCUMENT BEGINS HERE
%=-=-=-=-=-=-=-=-=-=-=-=-=-=-=-=-=-=-=-=-=-=-=-=-=-=-=-=-=-=-=-=-=-=-=-=-=-=-=-=
\begin{document}

% > > > If you are making use of the setspace package for line spacing then
%       you can choose one of the following options or refer to the setspace
%       documentation.
%\doublespacing%
%\onehalfspacing%

%=-=-=-=-=-=-=-=-=-=-=-=-=-=-=-=-=-=-=-=-=-=-=-=-=-=-=-=-=-=-=-=-=-=-=-=-=-=-=-=
% FRONTMATTER
%
% > > > By default the extended essay and internal assessment do not require any
%       frontmatter content.  It is here if you want/need it.
%=-=-=-=-=-=-=-=-=-=-=-=-=-=-=-=-=-=-=-=-=-=-=-=-=-=-=-=-=-=-=-=-=-=-=-=-=-=-=-=

%\frontmatter
%\renewcommand{\thepage}{\arabic{page}}

\maketitle

% > > > You can include non-chapter frontmatter content here.
%\include{chapters/frontmatter/preface}

% > > > Leaving some blank pages.
\cleardoublepage%

%=-=-=-=-=-=-=-=-=-=-=-=-=-=-=-=-=-=-=-=-=-=-=-=-=-=-=-=-=-=-=-=-=-=-=-=-=-=-=-=
% MAINMATTER
%
% > > > This is where your content will go.  It is probably best to break your
%       document up into chapters to make it easier to both navigate and resolve
%       errors in your code.
%=-=-=-=-=-=-=-=-=-=-=-=-=-=-=-=-=-=-=-=-=-=-=-=-=-=-=-=-=-=-=-=-=-=-=-=-=-=-=-=

\mainmatter%

%=-=-=-=-=-=-=-=-=-=-=-=-=-=-=-=-=-=-=-=-=-=-=-=-=-=-=-=-=-=-=-=-=-=-=-=-=-= TOC
\tableofcontents%

%=-=-=-=-=-=-=-=-=-=-=-=-=-=-=-=-=-=-=-=-=-=-=-=-=-=-=-=-=-=-=-=-=-=-=-=-=- PART
% > > > Your document probably does not need to be broken up into parts as
%       chapters should be more than sufficient to organize your work.
%\part{Not Necessary But Available}

%=-=-=-=-=-=-=-=-=-=-=-=-=-=-=-=-=-=-=-=-=-=-=-=-=-=-=-=-=-=-=-=-=-=-=- CHAPTERS
% > > > Here you have a choice to write you entire paper organized either
%       by chapter or by section.  However, if you do decide to use chapters,
%       then I would recommend using a separate file to organize your work.
%       I have included some demonstration chapters to help you.
% > > > Template Information
%=-=-=-=-=-=-=-=-=-=-=-=-=-=-=-=-=-=-=-=-=-=-=-=-=-=-=-=-=-=-=-=-=-= CHAPTER 01
\chapter{Introduction to IB\TeX}

% =-=-=-=-=-=-=-=-=-=-=-=-=-=-=-=-=-=-=-=-=-=-=-=-=-=-=-=-=-=-=-=-=-=-= SECTION
\section{Purpose}

This template is designed for International Baccalaureate\texttrademark\ 
students who are looking to write an extended essay or an internal assessment 
that is formatted using \LaTeX.  The template is based on the \KOMAScript\ scrbook
class and further customized using a number of custom style files, including  
\texttt{mhotext.sty}\footnote{\url{https://github.com/mholson/mhoDotFiles/texmf/}}
that is inspired by the code Evan Chan used create 
Napkin\footnote{\url{http://web.evanchen.cc/napkin.html}}. 

While I have done my best to ensure this template meets the submission
requirement of the IB, I cannot guarantee your document will meet the 
these standards.  It is the responsibility of the student to ensure that any 
work submitted to the IB for grading is formatted correctly by ensuring they 
are following the official IB documentation and the advice giving by their 
supervising teacher.

While there is extensive online support and documentation on how to publish 
using the \LaTeX\ programming language, it can still be overwhelming for anyone
who is getting started.  I have been navigating the \TeX\ ecosystem since 2005
and still learning.  I want to share some of the knowledge and ideas that I have
accumulated along the way, so that you can get writing
as quickly as possible.  You should be focused on writing an amazing 
manuscript and then put this template take care of the typesetting for you.

Before going any further, I feel that it is important that we establish that this 
documentation is written for high school students with little to no prior \TeX\ 
knowledge. While you should be able to typeset your document without an 
understanding of that \TeX\ is, how it works and a concise overview
of \TeX\ is and how it has developed over the past 40\textsuperscript{+} years,
I would recommend that you jump to chapter 06 before going any further
If you are not quite sure what the difference is between \LaTeX\ and \TeX , 
then I would highly recommend that you jump to chapter 06 for concise overview
of \TeX\ is and how it has developed over the past 40\textsuperscript{+} years.
While it is not imperative that you have an understanding of how \TeX\ has 
evolved, it does bring some useful context to how it works today.  
 

% =-=-=-=-=-=-=-=-=-=-=-=-=-=-=-=-=-=-=-=-=-=-=-=-=-=-=-=-=-=-=-=-=-=-= SECTION
\section{What is \LaTeX?}


so let's try to distill things down the necessities with a 
little historical context to guide us.  The 
programming language \TeX\ was created by Donal Knuth in the late
1970s to automate the process of formatting text ready for publication, 
a process called typesetting.  It worked by the user creating a \texttt{foo.tex} 
text file formatted with commands collectively called plain\TeX. This file would 
then be interpreted by a computer program called the \TeX\ engine, which would 
output a device independent file, \texttt{foo.dvi}, that could be sent to printer or 
converted to another file format such as \texttt{foo.ps} or \texttt{foo.pdf}.

However, in the 1980s Knuth strategically decided to that no new features would
be added to the \TeX engine as Knuth valued the benefits of an engine that
could produce the same output over new features being added. This has held true
up to today with the exception of one last feature being added in 1989 resulting
in \TeX3 \cite{GuideManyFlavours}.  With each bug fix update to \TeX3 the version number 
converges to \( \pi \) and in 2021 it reached version 3.14159265\textbf{3}\cite{CTANPackageLshort}.

Since \TeX\ is a programming language we are going to need a text editor
to write our code and a compiler to transform that code into a pdf document.
To keep things simple, I would highly recommend using  
Overleaf\footnote{\url{https://overleaf.com}}, which will provide you with  
both a modern text editor and a range of compilers that can all be used from
the comfort of your favorite web browser.  If you want to compile your code
on your own machine, then you will need to learn how to do that elsewhere.

Before you start coding, I would highly recommend that you spend some time
reading The Not So Short Introduction to \LaTeXe\ \cite{CTANPackageLshort}. 
Maybe start with at least the first two chapters just to give you some history
and context to \TeX.  Keep a link to this documentation close by as it 
makes for a great reference manual.

This document has been written so that you should be able to get into the 
weeds if you feel the need, but more importantly ensure that you can get 
writing almost immediately.  So now for a very generalized overview of \TeX.

You are probably familiar with the idea of writing your document all in one 
file using a word processor like MS Word or Google Documents. A shift in mindset
will be needed as your document will be now be an object called a project made 
up of multiple files and folders. 

\section{The Big Idea}

Basically, we are going to input code stored in a \texttt{main.tex} file into
a compiler that will generate an output file.  The code read by the compiler is 
written in using rhe written using the \TeX\ 
programming language, 

\section{Project Structure}
\begin{figure}[ht]

\dirtree{%
.1 \cnordOne{\faWarehouse} ~Project.
.2 \cnordThree{\faFileCode} ~main.tex.
.2 \cnordFifteen{\faFileCode} ~references.tex.
.2 \cnordEleven{\faFilePdf} ~main.pdf.
.2 \cnordTen{\faFolder} ~assets.
.3 \cnordTen{\faFolder} ~img.
.4 \cnordSeven{\faFileImage} ~exampleImage001.jpg.
.4 \cnordSeven{\faFileImage} ~exampleImage002.jpg.
.4 \cnordEleven{\faFilePdf} ~exampleImage003.pdf.
.2 \cnordTen{\faFolder} ~texmf.
.3 \cnordFourteen{\faFileCode} ~mhotext.sty.
.3 \cnordFourteen{\faFileCode} ~mhocolorthemenord.sty.
.3 \cnordFourteen{\faFileCode} ~mhomath.sty.
.3 \cnordFourteen{\faFileCode} ~mhocodestyle.sty.
.3 \cnordFourteen{\faFileCode} ~mhotheorems.sty.
.2 \cnordTen{\faFolder} ~chapters.
.3 \cnordThree{\faFileCode} ~chapter01.tex.
.3 \cnordThree{\faFileCode} ~chapter02.tex.
.3 \cnordThree{\faFileCode} ~chapter02.tex.
}
\caption{Example of a project file structure.}
\end{figure}


% =-=-=-=-=-=-=-=-=-=-=-=-=-=-=-=-=-=-=-=-=-=-=-=-=-=-=-=-=-=-=-=-=-=-= EXAMPLE
\begin{example}{Including an image.}
  When inserting an image into a MS Word document, you would insert the image 
  and it would become part of the document and you would no longer see that
  image file associated with your document.  
  
  When inserting an image into a \LaTeX ~document, we upload a copy of the 
  image into our project and then insert a line of code that will place that
  image into our document.
\end{example}

While the number of files can grow when writing a \LaTeX ~document, there are 
some pretty awesome advantages such as only having to modify or update an 
image file and the changes will be visible in your document following the 
next compilation of your code.

I am starting to get ahead of myself here. Let's 

It doesn't stop there!  All the template files necessary to format your document
will also exist in your project.  Don't worry, these files will be tucked away
in a folder called \texttt{texmf}.



% =-=-=-=-=-=-=-=-=-=-=-=-=-=-=-=-=-=-=-=-=-=-=-=-=-=-=-=-=-=-=-=-=-=-= SECTION

Our project is going to be made up of three folders and one main file called
\texttt{main.tex}. Let's describe the contents of those three folders.
\begin{description}
  \item[texmf] You probably will not do very much with the files located
        in this folder as these are the template files that are responsible for
        formatting the content of your essay. In most cases, you can change
        the formatting elements of your document right from the
        \texttt{main.tex} file.  It is also good practice to not edit these
        files directly, so that your theme can easily be upgraded if necessary.
  \item[assets] An asset is a file that you are going to include in your
        document.  In most cases, the only files that you will be including
        in your document will be images and thus you can upload your images
        directly to this folder.  You can have as many subfolders within your
        assets folder, but again this folder should be good enough to upload
        your images into.
  \item[chapters] This folder is really optional.  You could technically write
        your document all within the \texttt{main.tex} file; however, it might
        be beneficial to break your document up into chapters and
        write these in separate files.  If you are not going to be structuring
        your paper using chapters, then you should probably just write your
        document in the \texttt{main.tex} file.
\end{description}

Like I said before, I want to get you up and writing as quickly as possible.
We are going to get right into editing the \texttt{main.tex} file and producing
a pdf file with some content.  Action!

% =-=-=-=-=-=-=-=-=-=-=-=-=-=-=-=-=-=-=-=-=-=-=-=-=-=-=-=-=-=-=-=-=-=-= SECTION
\section{The Main File}
The \texttt{main.tex} file is where all the action takes place.  It is in this
file where you will write your content or at least link to the content found in
other files in your project.  This file can be as simple or as complicated as 
you would like it to be.  I have provided a skeleton template of this file in
your template project.  So let's start looking at the code.

For us the file is going to start of with multiple lines of text that are going
to be ignored by the compiler, which means it will not be included in our pdf
output.

% =-=-=-=-=-=-=-=-=-=-=-=-=-=-=-=-=-=-=-=-=-=-=-=-=-=-=-=-=-=-=-=-=-=-= SECTION
\section{texmf - The Template Files}

Most of you will never need to look in this folder; however, I thought I would
provide a short summary of what files are included and how they are being used
to format your document.

% =-=-=-=-=-=-=-=-=-=-=-=-=-=-=-=-=-=-=-=-=-=-=-=-=-=-=-=-=-=-=-=-=-=-= SECTION
\section{Requirements}

All the file dependencies outside the scope of a typical \TeX\, distribution
are available from my \texttt{texmf} repository hosted on
Github\footnote{\url{https://github.com/mholson/mhoTexmf/tree/main/texmf}}.
The Overleaf\texttrademark version will come with batteries included in the 
\texttt{texmf} directory, which in include.  
\begin{description}
    \item[\texttt{maabook.cls}] the main template file.
    \item[\texttt{mhotext.cls}] customizations made the to main template file.
    \item[\texttt{mhotheorems.sty}] styling custom theorem environments
    \item[\texttt{mhocolorthemenord}] custom color theme based on Nord Theme.
    \item[\texttt{mhocodestyle.sty}] styling code via the listings package.
    \item[\texttt{mhomath.sty}] some of my custom math commands.  
\end{description}


% =-=-=-=-=-=-=-=-=-=-=-=-=-=-=-=-=-=-=-=-=-=-=-=-=-=-=-=-=-=-=-=-=-=-= SECTION
\section{Getting Started}
% =-=-=-=-=-=-=-=-=-=-=-=-=-=-=-=-=-=-=-=-=-=-=-=-=-=-=-=-=-=-=-=-=- DEFINITION
\begin{definition}[Code Comment]
  Any code that is written in your file that is ignored by the compiler is
  called \defw{commented text}.  All characters on the same line following
  a \verb!%! symbol will be regarded as commented text and ignored.
\end{definition}

Comments are great for including additional information, making temporary 
changes to your document (without having to erase) and documenting your
code for others to follow --- including yourself!

\begin{dispExample}
% This line would be ignored by the compiler.
This line is read by the compiler.

The compiler is reading % and now it is ignoring.
\end{dispExample}
One of the benefits of writing in \LaTeX\, is the ability to write comments
that are not typeset.  Any text following a \verb!%! character will be 
ignored.  Not only is this ideal for leaving comments and ideas, but it is 
als a way to make temporary changes to your document.



\begin{mhotexbox}
% This code will be ignored
This text will be visible % while this text will be ignored.
\end{mhotexbox}
Much of your document will be filled with commands that take the form 
. In fact the first non-comment line of your document will be the command 
that defines the document template.
\begin{dispListing}
\documentclass[a4paper]{maatext}

\end{dispListing}
more text here

% > > > Mathematics
% =-=-=-=-=-=-=-=-=-=-=-=-=-=-=-=-=-=-=-=-=-=-=-=-=-=-=-=-=-=-=-=-=-= Chapter 02
\chapter{Typesetting Mathematics}

One of the main advantages of using \LaTeX\,is that it is relatively easy to
include mathematical statements into your documents.  In order for the code
that you write to be interpreted by the compiler as mathematics, you need
to escape from \defw{normal mode} into \defw{math mode} by using the
\verb!\(! delimiter, enter the code you wish to express as a mathematical
statement, then exit from \defw{math mode} back into \defw{normal mode} by
using the \verb!\)! delimiter. This \defw{math mode} is more specifically
known as inline mathematics.

% =-=-=-=-=-=-=-=-=-=-=-=-=-=-=-=-=-=-=-=-=-=-=-=-=-=-=-=-=-=-=-=-=-=-= SECTION
\section{Inline Mathematics}

Part of the steep learning curve associated with \LaTeX\,is getting to know the
math mode syntax.  We are going to start with an inline math example that
includes how to express a power, the command for is an element of the set and
how to format sets names using blackboard bold.

\begin{mhotexbox}
Solve all equations of the form \( ax^2 + bx + c = 0 \) for
\( x \in \mathbb{R} \) by completing the square.
\end{mhotexbox}

Spaces in \textbf{math mode} are ignored by the compiler, which means we can
be somewhat generous with space in our code to make it easier to parse.  I
generally like to leave space on both sides of:
\begin{enumerate}
\item explicit infix binary operators such as \( +, - , \times, \div \);
%\item binary relations such as \( = , \leq, \geq, \divides, >, < \);
  \item latex commands like \verb!\in, \mathbb{R}! and
  \item math mode toggle commands \verb!\(! and \verb!\)!.
\end{enumerate}
It is not required that you leave these spaces, but consider how cramped the
code looks without any spaces.
\begin{mhotexbox}
Solve all equations of the form \(ax^2+bx+c=0\) for
\(x\in\mathbb{R}\) by completing the square.
\end{mhotexbox}
As your expressions become more complex, the additional spaces in your code
should improve the legibility of your code.  Well, at least that has been my
experience. If you are new to \LaTeX\,, then you might want to use the
plain \TeX\,\verb!$! delimiters to toggle between \textbf{normal mode} and
\textbf{math mode}.  The advantage is that it is easier to read the code, but
can make finding errors more difficult to locate later on.  You can always
do a search replace to go from \verb!\( \)! to \verb!$$!, but the converse
is not possible.
\begin{mhotexbox}
Solve all equations of the form $ ax^2 + bx + c = 0 $ for $ x \in \mathbb{R} $
by completing the square.
\end{mhotexbox}
It is now time to introduce \defw{display style math mode}.

% =-=-=-=-=-=-=-=-=-=-=-=-=-=-=-=-=-=-=-=-=-=-=-=-=-=-=-=-=-=-=-=-=-=-= SECTION
\section{Display Style Math Mode}

There will be times when you will want to emphasise or isolate an expression
by centering it on the page and for this we are going to be using 
\defw{display style math mode}.  If your expression is only one line then
we can use the \verb!\[ foo \]! delimiters, which can formatted such that
each delimiter gets its own line and the expression is given on the line 
between them.  You will most likely come across the plain \TeX\,double
dollar sign, \verb!$$ foo $$!, delimiters and unlike the single dollar sign
as delimiters in inline mode, these are \textbf{NOT TO BE USED} to toggle
display style math mode.

\begin{mhotexbox}
All the real number values of \( x \) that satisfy the quadratic equations
of the form \( ax^2 + bx + c = 0 \) where \( a, b, c \in \mathbb{R} \) 
can be found using the quadratic formula,
\[
  x = \frac{-b \pm \sqrt{b^2 - 4ac}}{2a}.
\]
\end{mhotexbox}

If you are looking to include multiple lines in display math mode, then
you have a few environment options available.  You should not use 
consecutive display math mode delimiters.  The following is to be avoided.

\begin{mhotexbox}
  \[ 
    2x^2 + 14x + 24 = 2( x^2 + 7x + 12 )
  \]
  \[ 
    = 2(x + 3)(x + 3) 
  \].
\end{mhotexbox}

There are many different environments for doing mutliple line expressions
in \defw{display math mode}. I use the \texttt{align} environment for
about 95 \% of my mutliple line expressions needed and is dependent upon 
the amsmath package.  I really like how I can format my code such that 
each expression of the equation \textbf{can} be given its own line, 
which is very useful for finding errors and viewing long lines of code.
In the upcomming example you will see that I use the \verb!&! to align
each of the multiple lines with the \verb!=! sign and use \verb!\\!
for an equation line break.

\begin{mhotexbox}
Changing the form of the expression \( 2x^2 + 14x + 24 \) in inline math 
mode, could write \( 2x^2 + 14x + 24 = 2( x^2 + 7x + 12 ) = 2(x + 3)(x + 3) \).  
However, this might be easier for the reader if it was expressed in 
display math mode using the align environment,
\begin{align}
  2x^2 + 14x + 24 
  &= 2( x^2 + 7x + 12 ) \label{eqn:importantInfo}\\
  &= 2(x + 3)(x + 3)  \label{eqn:finalAnswer}
\end{align}
Expression \ref{eqn:finalAnswer} is my final answer.
\end{mhotexbox}
You probably noticed that each line of this environment was assigned a 
number. If you want to be able to reference a line from the environemnt 
you can provide it with a label and then reference it as 
expression \ref{eqn:importantInfo}.

It might be the case that you do not want to reference or number each 
expression in an align environment.  In this case, you can use the 
can append a star to the environment name to suppress the numbering.

\begin{mhotexbox}
  Given \( x = \cos x \) and \( y = \sin x \), we can show
  \begin{align*}
    x^2 + y^2 &= 1 \\
    \left( \cos x \right)^2 + \left( \sin x \right)^2 &= 1 \\
    \cos^2 x + \sin^2 &= 1.
  \end{align*}
\end{mhotexbox}

It is also possible to provide text between each line of your 
environment without having to use a new environment each time 
you want to move back to display style math mode.  This is 
the \verb!intertext! command.  If you want less space between
each line then you can use the \verb!shortintertext! command.

\begin{mhotexbox}
  Given \( x = \cos x \) and \( y = \sin x \), we can show
  \begin{align*}
    \shortintertext{using the Pythagorean identity}
    x^2 + y^2 &= 1 \\
    \intertext{substituting \(x \) and \(y \)}
    \left( \cos x \right)^2 + \left( \sin x \right)^2 &= 1 \\
    \shortintertext{alternative function squared form}
    \cos^2 x + \sin^2 &= 1.
  \end{align*}
\end{mhotexbox}

% > > > Figure & Image Elements
%=-=-=-=-=-=-=-=-=-=-=-=-=-=-=-=-=-=-=-=-=-=-=-=-=-=-=-=-=-=-=-=-=-=- CHAPTER 03
\chapter{Images, Figures \& Tables}

%=-=-=-=-=-=-=-=-=-=-=-=-=-=-=-=-=-=-=-=-=-=-=-=-=-=-=-=-=-=-=-=-=-=-=-= SECTION
\section{Storing Images}

Images can be easily included in your document by using the \cs{includegraphics}
command included with the graphics package. Before we include our first image, we 
should probably create a folder to hold all our images to keep our folder 
lean in terms of the number of files.  My personal preference is to keep all my 
images in the folder called \verb!~/assets!, which is relative to where the 
main file is located - the root of the project folder. If you are looking for 
a more refined organization of your images, then it is also possible to 
populate your assets folder with subfolders.

When we reference an image file, we need to provide both the path and file name.
Fortunately, there is a way to assign the path(s) where you store your images,
so you do not have to include the path to each image.  This can be done by 
including the \cs{graphicspath} command in the preamble of the script to assign
the path of where your images are stored.
\begin{dispListing}
\graphicspath{{/assets/}}
\end{dispListing}
It is also possible to define more than one path, which you might want to do 
if you are using subfolders to organize your images.  Just be careful to 
ensure all your images have unique names if you do decide to store your 
images in multiple folders.
\begin{dispListing}
\graphicspath{{/assets/images}{/assets/pdfs/}{/assets/matplotlib/}{/assets/tikz/}}
\end{dispListing}
This command will save you a lot of typing and frustration, so I would 
encourage you to try it.

%=-=-=-=-=-=-=-=-=-=-=-=-=-=-=-=-=-=-=-=-=-=-=-=-=-=-=-=-=-=-=-=-=-=-=-= SECTION
\section{Images}

To include an image into our document, we will use the \cs{includegraphics} 
command, which accepts the path and file name as its argument and can take 
on multiple options.  If you made use of the \cs{graphicspath} command made 
available by default for this template, then the argument is simplify the file 
name.  The file extension is optional, unless you have multiple files with the 
same name, but with different extensions. Yes, \verb!foo.pdf! are valid 
image files to include into your document. 
\begin{dispListing}
    \includegraphics{myimage.jpg}
    \includegraphics{myimage.pdf}
\end{dispListing}
The option that I set most often when using the \cs{includegraphics}
command is the width of the image.  For example, let's say that I want to 
include an image called \verb!image.pdf! into my document such that it is 
centered on the page and that the width of the image is set to a quarter of 
the text width of the page. 
\begin{dispExample}
\begin{center}
\includegraphics[width=0.25\textwidth]{image}
\end{center}
\end{dispExample}
Just be careful when working with raster images such as \verb!foo.jpg! as 
these types of images do not scale well.  I try to ensure that all my images 
are vector based such as a \verb!foo.pdf!, which allows for beautiful scaling.

% TODO Include height scale
% TODO Include scale
% TODO Include rotate 

%=-=-=-=-=-=-=-=-=-=-=-=-=-=-=-=-=-=-=-=-=-=-=-=-=-=-=-=-=-=-=-=-=-=-=-= SECTION
\section{Figures}

The \XeLaTeX\ engine is typesetting the document for us and placing images on 
the page is part of that process.  An inherent problem with images is that they 
cannot be broken across two pages like a paragraph of text.  A common
solution is to create a page break each time you have an image that spans over 
two pages.  Let's leave the typesetting to the engine because let's face it,
if you have many images that are spanning two pages, your document is not going 
to look good with all that extra whitespace created by forcing those page breaks.
The solution is a floating environment called a figure to embed your image. 

A float works by moving the image that does not fit on the page to a later one
and fill up the white space of the current page with text.  It is time to 
let go of implicitly referencing images by their placement of the document. 
We should now be referencing our images and let the engine calculate the best placement
of all our floats within our document.  It might feel better knowing that 
you can give your image a caption to provide some context as it might no 
longer be related to the surrounding text.  Just be ready for the engine to
never place the float where you would have placed it.

\begin{docEnvironment*}[doclang/environment content=includegraphics here]{figure}{\oarg{!htbp}}{}
\end{docEnvironment*}

You can see that the figure environment includes some optional arguments, called 
placement specifiers\cite{CTANPackageLshort}, that give you some control over where
the engine will place the float. The order in which the placement specifiers
are written is the priority the engine will give to placing your float within your 
documentation.  If no placement identifiers are given, then it will default to \oarg{tbp}. 
See table \cref{tbl:placementspecifiers} for a summary which each optional
argument does.
\begin{table}
\begin{tblr}{c X[l]}
\hline
\oarg{h} & Places the float \textbf{here} relative to the text you have written.
This should only be used for small floats. \\
\oarg{t} & Place the float at the \textbf{top} of the next available page.\\
\oarg{b} & Place the float at the \textbf{bottom} of the next available page. \\
\oarg{p} & Place on a \textbf{page} only containing floats. \\
\oarg{!} & Force one of the above options even if it does not look good.\\
\hline
\end{tblr}
\label{tbl:placementspecifiers}
\caption{Float Placement Specifiers}
\end{table}
You will probably want to reference your image at some point in your text and
will most likely want to give it a caption.  For this we are going to
use the figure environment, which will calculate the best position to place 
your image within the document.  Don't worry, there are ways to manually place 
the image where you would like it to appear.
\begin{docEnvironment}[doclang/environment content=image content goes here]{figure}{\oarg{h, t}}{}
    \begin{dispListing}
        \begin{center}
        \includegraphics[width = 0.25\textwidth]{image.jpg}
        \end{center}
        \caption{Here is my excellent image.}
        \label{fig:img002}
        \end{dispListing}
\end{docEnvironment}
\begin{figure}[h]
    \begin{center}
    \includegraphics[width = 0.5\textwidth]{image.jpg}
    \end{center}
    \caption{Here is my excellent image.}
    \label{fig:img001}
\end{figure}
This would generate figure \ref{fig:img001}.

%=-=-=-=-=-=-=-=-=-=-=-=-=-=-=-=-=-=-=-=-=-=-=-=-=-=-=-=-=-=-=-=-=-=-=-= SECTION
\section{Multiple Horizontally Aligned Images}

Quite often it is the case that we want to include multiple horizontally
aligned images.  The figure environment is not well equipped for this task, so
we care going to make use of the subfigure package which will give us the 
subfigure environment that we can embed in our figure environment.  This will
enable us to assign both a caption and a label to each image separately.
\begin{docEnvironment}[doclang/environment content=image content goes here]{subfigure}{}{}
\begin{dispListing}
\begin{figure}
    \begin{center}
    \begin{subfigure}[b]{0.3\textwidth}
        \includegraphics[width=\textwidth]{20170509-123642}
        \caption{Step 1}
        \label{fig:step1a}
    \end{subfigure}
    ~   % > > > Add desired spacing between images, e. g. ~, \quad, \qquad,
                % \hfill etc. (or a blank line to force the subfigure onto
                % a new line)
    \begin{subfigure}[b]{0.3\textwidth}
        \includegraphics[width=\textwidth]{20170509-123642.png}
        \caption{Step 2}
        \label{fig:step2a}
    \end{subfigure}
    ~   % > > > Add desired spacing between images, e. g. ~, \quad, \qquad,
                % \hfill etc. (or a blank line to force the subfigure onto
                % a new line)
    \begin{subfigure}[b]{0.3\textwidth}
        \includegraphics[width=\textwidth]{20170509-123642.png}
        \caption{Step 3}
        \label{fig:step3a}
    \end{subfigure}
    \caption{The Steps Shown}\label{fig:threestepsa}
    \end{center}
\end{figure}
\end{dispListing}
\end{docEnvironment}
\begin{figure}
    \begin{center}
    \begin{subfigure}[b]{0.3\textwidth}
        \includegraphics[width=\textwidth]{20170509-123642}
        \caption{Step 1}
        \label{fig:step1}
    \end{subfigure}
    ~   % > > > Add desired spacing between images, e. g. ~, \quad, \qquad,
                % \hfill etc. (or a blank line to force the subfigure onto
                % a new line)
    \begin{subfigure}[b]{0.3\textwidth}
        \includegraphics[width=\textwidth]{20170509-123642.png}
        \caption{Step 2}
        \label{fig:step2}
    \end{subfigure}
    ~   % > > > Add desired spacing between images, e. g. ~, \quad, \qquad,
                % \hfill etc. (or a blank line to force the subfigure onto
                % a new line)
    \begin{subfigure}[b]{0.3\textwidth}
        \includegraphics[width=\textwidth]{20170509-123642.png}
        \caption{Step 3}
        \label{fig:step3}
    \end{subfigure}
    \caption{The Steps Shown}\label{fig:threesteps}
    \end{center}
\end{figure}

%=-=-=-=-=-=-=-=-=-=-=-=-=-=-=-=-=-=-=-=-=-=-=-=-=-=-=-=-=-=-=-=-=-=-=-= SECTION
\section{Tables}

It is very likely that you will want to include tables in your document.  
Tables are not always easy to generate using \LaTeX ; however the tabularray 
package makes formatting tables a little easier then some of the more 
popular table packages such as tabularx.  Tabularray is well documented and
supports tables in both text and math modes.  The package and custom styles 
used for tables are included in the mhotables.sty custom package.  This makes 
it easy for you to use your favorite tables package by only having to modify
one line of code in the mhotext.sty.  Of course, by doing so this documentation 
document will no longer compile as it is dependent on mhotables.sty.

\begin{dispListing}
%\RequirePackage{mhotables.sty}
\RequirePackage{tabularx}
\end{dispListing}

The table itself is placed in a table environment, which will allow the 
\XeLaTeX\ engine to calculate the best place for the table within your
document very much the same way the figure environment works.  It will
also enable to give your table a descriptive label using the \cs{caption}
command and give it a label so that you can reference it in your document. 
I usually propend tbl: to my table captions to make them easier to look up.

\begin{dispListing}
\begin{table}
    \begin{center}
    \begin{tblr}
    % Code to format the table
    \end{tblr}
\end{center}
\label{tbl:myfirsttable}
\caption{This table will be included in the document soon.}
\end{table}
\end{dispListing}

\begin{dispExample}
\begin{center}
\(\begin{tblr}[ 
    caption = {These display fractions look awesome out of the box!},
    label = {tblr:fractions}
]{rrr}
\hline
\dfrac{2}{3} &  \dfrac{2}{3} &  \dfrac{1}{3} \\
\dfrac{2}{3} & -\dfrac{1}{3} & -\dfrac{2}{3} \\
\dfrac{1}{3} & -\dfrac{2}{3} &  \dfrac{2}{3} \\
\hline
\end{tblr} \)
\end{center}
\end{dispExample}

Another example in math mode, would be to make use of the \cs{diagbox}
command which has been enabled by including \cs{UseTblrLibrary{diagbox}}
in the preamble of the style file.  Don't worry this is included by 
default.

\begin{dispExample}
    \begin{center}
    \(\begin{tblr}{|c||c|c|c|c|c|c||}
    \hline
    \diagbox{f(x)}{x}   &   0   &   1   &   2   &   3   &   4   &   5\\
    \hline
    x^2                 &   0   &   \SetCell{bg=nordEleven,fg=nordFour} \smile  &   3   &   9   &   16  &   25 \\
    \hline
    \end{tblr} \)
    \end{center}
    \end{dispExample}

\begin{table}
    \caption{hello world}
    \begin{center}
    \(\begin{tblr}{|c||c|c|c|c|c|c|}
    \hline
    \diagbox{f(x)}{x}   &   \cpoint   &   \interval   &   2   &   3   &   4   &   5\\
    \hline
    x^2                 &   \cellTeal \checkcirc   &  \cellRed \decf &   \cellGrey \stationary  &   \cellGreen \incf   &   \cellTeal +  &   \cellOrange - \\
    \hline
    \end{tblr} \)
    \end{center}
\end{table}


%%% Local Variables:
%%% mode: latex
%%% TeX-master: "../main"
%%% End:

% > > > Theorem Environments
%%=-=-=-=-=-=-=-=-=-=-=-=-=-=-=-=-=-=-=-=-=-=-=-=-=-=-=-=-=-=-=-=-=-=- CHAPTER 03
\chapter{Theorem Environments}

%=-=-=-=-=-=-=-=-=-=-=-=-=-=-=-=-=-=-=-=-=-=-=-=-=-=-=-=-=-=-=-=-=-=-=-= SECTION
\section{Definition}

%=-=-=-=-=-=-=-=-=-=-=-=-=-=-=-=-=-=-=-=-=-=-=-=-=-=-=-=-=-=-=-==-=-= DEFINITION
\begin{definition}[Definition Of Subtraction]
\label{0000}\index{Definition!0000}
Given \( a \) and \( b \) are real numbers, then
\cite{Olson2021}
\begin{align*}
    a + (-b) &= a - b.
\end{align*}
\end{definition}
%=-=-=-=-=-=-=-=-=-=-=-=-=-=-=-=-=-=-=-=-=-=-=-=-=-=-=-=-=-=-=-=- END DEFINITION

%=-=-=-=-=-=-=-=-=-=-=-=-=-=-=-=-=-=-=-=-=-=-=-=-=-=-=-=-=-=-=-=-=-=-=-= SECTION
\section{Examples}

%=-=-=-=-=-=-=-=-=-=-=-=-=-=-=-=-=-=-=-=-=-=-=-=-=-=-=-=-=-=-=-=-=-=-=-= EXAMPLE
\begin{example}
\label{0001}\index{Example!0001}
Solve the equation \( x + 4 = 7 \) for all \( x \in \setZ \).
\end{example}
% solution
\begin{solution}
The solution set is \( x = \set{3} \).
\end{solution}
%=-=-=-=-=-=-=-=-=-=-=-=-=-=-=-=-=-=-=-=-=-=-=-=-=-=-=-=-=-=-=-=-=-= END EXAMPLE

%=-=-=-=-=-=-=-=-=-=-=-=-=-=-=-=-=-=-=-=-=-=-=-=-=-=-=-=-=-=-=-=-=-=-=-= SECTION
\section{Theorems}

%=-=-=-=-=-=-=-=-=-=-=-=-=-=-=-=-=-=-=-=-=-=-=-=-=-=-=-=-=-=-=-==-=-=-=-=- AXIOM
\begin{axiom}[Axiom Of Simple Mathematical Induction]
\label{0005}\index{axiom!0005}
Let \(P(n)\) be a proposition, where \( n \in \setZp \) .  If we can
\begin{enumerate}
    \item show the basis step is TRUE by showing \(P(1)\) is TRUE and 
    \item show the inductive step is TRUE by showing for each 
    \( r \in \setZp \), whenever \( P(r) \) is TRUE , then \( P(r + 1) \) is
    TRUE, where \( P(r) \) is called the \defw{induction hypothesis}, 
\end{enumerate}
then by the \defw{axiom of mathematical induction} the proposition is proved.
        
The axiom of simple mathematical induction cannot be proven as it is part of the definition of \(\setZp\).
        
\end{axiom}
%=-=-=-=-=-=-=-=-=-=-=-=-=-=-=-=-=-=-=-=-=-=-=-=-=-=-=-=-=-=-=-==-=-=- END AXIOM

The axiom of simple mathematical induction\ref{0005} cannot be proven as it is 
included as part of the definition of the set of positive integers, 
\( \setZp \).

%=-=-=-=-=-=-=-=-=-=-=-=-=-=-=-=-=-=-=-=-=-=-=-=-=-=-=-=-=-=-=-==-=- PROPOSITION
\begin{proposition}[Product of Common Base Powers]
\label{0003}\index{Proposition!0003}
Given \( b^m \) and \( b^n \), then
\begin{align*}
    b^m \cdot b^n &= b^{m + n}.
\end{align*}
\end{proposition}
\begin{proof}
    Coming Soon!
\end{proof}
%=-=-=-=-=-=-=-=-=-=-=-=-=-=-=-=-=-=-=-=-=-=-=-=-=-=-=-=-=-=-=-= END PROPOSITION

%=-=-=-=-=-=-=-=-=-=-=-=-=-=-=-=-=-=-=-=-=-=-=-=-=-=-=-=-=-=-=-==-=- PROPOSITION
\begin{theorem}[Pythagorean Theorem]
    \label{0004}\index{Theorem!0004}
    Given a right-angled triangle \( ABC \) with sides \( a, b \) and \( c \),
    where \( c \) is the hypotenuse, then
    \begin{align*}
        a^2 + b^2 &= c^2.
    \end{align*}
    \end{theorem}
    \begin{proof}
        Someday soon!
    \end{proof}
%=-=-=-=-=-=-=-=-=-=-=-=-=-=-=-=-=-=-=-=-=-=-=-=-=-=-=-=-=-=-=- END PROPOSITION


% > > > Works Cited
%%=-=-=-=-=-=-=-=-=-=-=-=-=-=-=-=-=-=-=-=-=-=-=-=-=-=-=-=-=-=-=-=-=-=- CHAPTER 03
\chapter{Works Cited}


% > > > What is LaTeX
%%=-=-=-=-=-=-=-=-=-=-=-=-=-=-=-=-=-=-=-=-=-=-=-=-=-=-=-=-=-=-=-=-=-=- CHAPTER 06
\chapter{What is \LaTeX?}

% =-=-=-=-=-=-=-=-=-=-=-=-=-=-=-=-=-=-=-=-=-=-=-=-=-=-=-=-=-=-=-=-=-=-= SECTION
\section{A Brief History of \TeX}

The programming language \TeX\ was created by Donald Knuth in the late
1970s to typeset beautifully formatted documents that could be easily published
using a computer without having to go through professionals typesetters.
Knuth was trying to solve the problem of having to bring unformatted writing,
called a manuscript, to a typesetter where they would decide how to best format
your work by deciding things such as the font, how to space the characters on a
line and where to place images. Imagine how stressed you would feel sending off
a piece of IB coursework to be typeset and then have it returned in a form
that did not meet your expectations and being forced to sent that published
work to be graded.

Today, desktop publishing has never been easier with options such as Microsoft
Word and Google Documents.  So why has \TeX\ remained the standard in the
mathematics and scientific community for writing and publishing documents?
With these applications being easy to use, there must be some compelling
reasons to invest in learning a computer language that has a steep learning
curve.

As you probably guessed, the typesetter probably will not format your document 
exactly how you envisioned it to be.  You can imagine the frustration of having 
to give feedback on the document produced by the typesetter and then wait for their 
modifications. 

the automate the process of formatting text ready for publication, 
a process called typesetting.  It seems that Knuth was inspired to take control
of the typesetting process in 1974 when he stopped sending his work
the the American Mathematics Society because \textquote[{\cite{FoRKArchiveBrief}}]{the finished
product was just too painful for me to look at}.  On May 13, 1977, Knuth 
shares a description of his preliminary vision of how \TeX\ will work to automate 
automate the process of formatting text ready for publication

It worked by the user creating a \texttt{foo.tex} 
text file formatted with commands collectively called plain\TeX\. This file would 
then be interpreted by a computer program called the \TeX\ engine, which would 
output a device independent file, \texttt{foo.dvi}, that could be sent to printer or 
converted to another file format such as \texttt{foo.ps} or \texttt{foo.pdf}.
%=-=-=-=-=-=-=-=-=-=-=-=-=-=-=-=-=-=-=-=-=-=-=-=-=-=-=-=-=-=-=-=-=-=- DEFINITION
\begin{definition}[Engine]
\label{defn:engine}\index{Definition!Engine}
An \defw{engine} is a computer program that \defw{compiles} one or more input 
files written using a programming language to produce a output file.
\end{definition}
%=-=-=-=-=-=-=-=-=-=-=-=-=-=-=-=-=-=-=-=-=-=-=-=-=-=-=-=-=-=-=-=- END DEFINITION

% =-=-=-=-=-=-=-=-=-=-=-=-=-=-=-=-=-=-=-=-=-=-=-=-=-=-=-=-=-=-=-=-=-=-= SECTION
\section{A Brief History of \TeX}



% > > > Appendix
\clearpage
\part{Appendix}
\appendix
%\include{chapters/somechapter}
%=-=-=-=-=-=-=-=-=-=-=-=-=-=-=-=-=-=-=-=-=-=-=-=-=-=-=-=-=-=-=-=-=-=-=-=-=-=-=-=
% BACKMATTER
%
% > > > Here you will include your reference section and possibly appendices if you
%       feel that you do not want to add them to the frontmatter
%=-=-=-=-=-=-=-=-=-=-=-=-=-=-=-=-=-=-=-=-=-=-=-=-=-=-=-=-=-=-=-=-=-=-=-=-=-=-=-=
\backmatter%
%=-=-=-=-=-=-=-=-=-=-=-=-=-=-=-=-=-=-=-=-=-=-=-=-=-=-=-=-=-=-=-=-=-= REFERENCES
\clearpage
\printbibliography[title={References}]
\end{document}
%=+=+=+=+=+=+=+=+=+=+=+=+=+=+=+=+=+=+=+=+=+=+=+=+=+=+=+=+=+=+=+=+=+=+=+=+=+=+=+=
% END OF FILE
% =+=+=+=+=+=+=+=+=+=+=+=+=+=+=+=+=+=+=+=+=+=+=+=+=+=+=+=+=+=+=+=+=+=+=+=+=+=+=+=

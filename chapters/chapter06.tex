%=-=-=-=-=-=-=-=-=-=-=-=-=-=-=-=-=-=-=-=-=-=-=-=-=-=-=-=-=-=-=-=-=-=- CHAPTER 06
\chapter{What is \LaTeX?}

% =-=-=-=-=-=-=-=-=-=-=-=-=-=-=-=-=-=-=-=-=-=-=-=-=-=-=-=-=-=-=-=-=-=-= SECTION
\section{A Brief History of \TeX}

The programming language \TeX\ was created by Donald Knuth in the late
1970s to typeset beautifully formatted documents that could be easily published
using a computer without having to go through professionals typesetters.
Knuth was trying to solve the problem of having to bring unformatted writing,
called a manuscript, to a typesetter where they would decide how to best format
your work by deciding things such as the font, how to space the characters on a
line and where to place images. Imagine how stressed you would feel sending off
a piece of IB coursework to be typeset and then have it returned in a form
that did not meet your expectations and being forced to sent that published
work to be graded.

Today, desktop publishing has never been easier with options such as Microsoft
Word and Google Documents.  So why has \TeX\ remained the standard in the
mathematics and scientific community for writing and publishing documents?
With these applications being easy to use, there must be some compelling
reasons to invest in learning a computer language that has a steep learning
curve.

As you probably guessed, the typesetter probably will not format your document 
exactly how you envisioned it to be.  You can imagine the frustration of having 
to give feedback on the document produced by the typesetter and then wait for their 
modifications. 

the automate the process of formatting text ready for publication, 
a process called typesetting.  It seems that Knuth was inspired to take control
of the typesetting process in 1974 when he stopped sending his work
the the American Mathematics Society because \textquote[{\cite{FoRKArchiveBrief}}]{the finished
product was just too painful for me to look at}.  On May 13, 1977, Knuth 
shares a description of his preliminary vision of how \TeX\ will work to automate 
automate the process of formatting text ready for publication

It worked by the user creating a \texttt{foo.tex} 
text file formatted with commands collectively called plain\TeX\. This file would 
then be interpreted by a computer program called the \TeX\ engine, which would 
output a device independent file, \texttt{foo.dvi}, that could be sent to printer or 
converted to another file format such as \texttt{foo.ps} or \texttt{foo.pdf}.
%=-=-=-=-=-=-=-=-=-=-=-=-=-=-=-=-=-=-=-=-=-=-=-=-=-=-=-=-=-=-=-=-=-=- DEFINITION
\begin{definition}[Engine]
\label{defn:engine}\index{Definition!Engine}
An \defw{engine} is a computer program that \defw{compiles} one or more input 
files written using a programming language to produce a output file.
\end{definition}
%=-=-=-=-=-=-=-=-=-=-=-=-=-=-=-=-=-=-=-=-=-=-=-=-=-=-=-=-=-=-=-=- END DEFINITION

% =-=-=-=-=-=-=-=-=-=-=-=-=-=-=-=-=-=-=-=-=-=-=-=-=-=-=-=-=-=-=-=-=-=-= SECTION
\section{A Brief History of \TeX}


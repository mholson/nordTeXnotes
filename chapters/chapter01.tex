%=-=-=-=-=-=-=-=-=-=-=-=-=-=-=-=-=-=-=-=-=-=-=-=-=-=-=-=-=-=-=-=-=-=- CHAPTER 01
\chapter{Introduction}

%=-=-=-=-=-=-=-=-=-=-=-=-=-=-=-=-=-=-=-=-=-=-=-=-=-=-=-=-=-=-=-=-=-=-=-= SECTION
\section{Purpose}

This template is designed for International Baccalaureate\texttrademark\, 
students who are looking to write an extended essary or an internal assessment
using \LaTeX.  It is based on the MAA monograph AMS/MAA textbook template
\footnote{\url{http://www.ams.org/arc/books/book-produce.html}} class files 
\verb!maatext.cls! and \verb!maabook.cls! which have been modifed to try and
meets the formatting requirements and recommendations of the International 
Baccalaureate\texttrademark\, These modifications and additional file 
dependencies are maintained in github texmf repository.
\footnote{\url{https://github.com/mholson/mhoTexmf/tree/main/texmf}}.  

I have done my best to ensure this template meets the submission requirements; 
however, I cannot guarentee these are current nor correct. It is the 
responsibility to the student to ensure the work they submit is formatted 
correctly by ensuring they are following the official documentation and
following the advice of their teacher.\cite{Olson2021}

\begin{align*}
    ax^2 + bx + c &= 0 \\
    \cos^2 x + \sin^2 x &= 0 \\
    \ln x + \dfrac{2x^2}{y} &= \rfrac{3}{x}
\end{align*}

While there is extensive online support and documentation for \LaTeX\,, it 
can still be overwhelming to a beginner.  I do hope the following information 
and examples will give students a sufficient knowledge base from which they 
can beging writing and extend their skills. 

%=-=-=-=-=-=-=-=-=-=-=-=-=-=-=-=-=-=-=-=-=-=-=-=-=-=-=-=-=-=-=-=-=-=-=-= SECTION
\section{Requirements}

%=-=-=-=-=-=-=-=-=-=-=-=-=-=-=-=-=-=-=-=-=-=-=-=-=-=-=-=-=-=-=-=-=-=-=-= SECTION
\section{Getting Started}

One of the benfits of writing in \LaTeX\, is the ability to write comments
that are not typeset.  Any text following a \lstinline{%} character will be 
ignored.  Not only is this ideal for leaving comments and ideas, but it is 
als a way to make temporary changes to your document.
\lstset{style=mhotexcode}
\begin{lstlisting}[belowskip=-2 \baselineskip]
% This code will be ignored
This text will be visible % while this text will be ignored.
\end{lstlisting}
Much of your document will be filled with commands that take the form 
. In fact the first non-comment l
ine of your document will be the command that defines the document template.
\begin{lstlisting}
\documentclass[a4paper]{maatext}
\end{lstlisting}

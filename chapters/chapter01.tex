%=-=-=-=-=-=-=-=-=-=-=-=-=-=-=-=-=-=-=-=-=-=-=-=-=-=-=-=-=-=-=-=-=-=- CHAPTER 01
\chapter{Introduction}

%=-=-=-=-=-=-=-=-=-=-=-=-=-=-=-=-=-=-=-=-=-=-=-=-=-=-=-=-=-=-=-=-=-=-=-= SECTION
\section{Purpose}

This template is designed for International Baccalaureate\texttrademark\, 
students who are looking to write an extended essay or an internal assessment 
using \LaTeX.  The template is based on the KOMA Script scrbook class and styled
using the custom style file 
\texttt{mhotext.sty}\footnote{\url{https://github.com/mholson/mhoDotFiles/texmf/}}, 
which was inspired by Evan Chan's code
used create Napkin\footnote{\url{http://web.evanchen.cc/napkin.html}}.

\begin{align*}
    ax^2 + bx + c &= 0 \\
    \cos^2 x + \sin^2 x &= 0 \\
    \ln x + \dfrac{2x^2}{y} &= \rfrac{3}{x}
\end{align*}

While there is extensive online support and documentation for \LaTeX\,, it 
can still be overwhelming to a beginner.  I do hope the following information 
and examples will give students a sufficient knowledge base from which they 
can beging writing and extend their skills. 

%=-=-=-=-=-=-=-=-=-=-=-=-=-=-=-=-=-=-=-=-=-=-=-=-=-=-=-=-=-=-=-=-=-=-=-= SECTION
\section{Requirements}

%=-=-=-=-=-=-=-=-=-=-=-=-=-=-=-=-=-=-=-=-=-=-=-=-=-=-=-=-=-=-=-=-=-=-=-= SECTION
\section{Getting Started}

One of the benfits of writing in \LaTeX\, is the ability to write comments
that are not typeset.  Any text following a \lstinline{%} character will be 
ignored.  Not only is this ideal for leaving comments and ideas, but it is 
als a way to make temporary changes to your document.
\lstset{style=mhotexcode}
\begin{lstlisting}[belowskip=-2 \baselineskip]
% This code will be ignored
This text will be visible % while this text will be ignored.
\end{lstlisting}
Much of your document will be filled with commands that take the form 
. In fact the first non-comment l
ine of your document will be the command that defines the document template.
\begin{lstlisting}
\documentclass[a4paper]{maatext}
\end{lstlisting}

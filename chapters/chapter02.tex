% =-=-=-=-=-=-=-=-=-=-=-=-=-=-=-=-=-=-=-=-=-=-=-=-=-=-=-=-=-=-=-=-=-= Chapter 02
\chapter{Typesetting Mathematics}

%=-=-=-=-=-=-=-=-=-=-=-=-=-=-=-=-=-=-=-=-=-=-=-=-=-=-=-=-=-=-=-=-=-=-=-= SECTION
\section{Inline Mathematics}
Any mathematics that is to be typset inline with text is called 
\defw{inline math}.  

%=-=-=-=-=-=-=-=-=-=-=-=-=-=-=-=-=-=-=-=-=-=-=-=-=-=-=-=-=-=-=-=-=-=-=-= EXAMPLE
\begin{example}
It is time that we solve all equations of the form \(ax^2 + bx + c = 0 \) by 
completing the square.  

There are two acceptable notations for typsetting the equation.  
\begin{description}
    \item[TeX] \verb! $ ax^2 + bx + c = 0 $ !
    \item[LaTeX] \verb! \( ax^2 + bx + c = 0 \) !
\end{description}

\end{example}
%=-=-=-=-=-=-=-=-=-=-=-=-=-=-=-=-=-=-=-=-=-=-=-=-=-=-=-=-=-=-=-=-=-= END EXAMPLE

As a beginner, it might be easier to use the \verb!$! to delineate any inline
math.  I personally prefer using the \LaTeX\, notation.  It is probably  
worth mentioning that all whitespace is ignored for inline math, so you 
might want to create some style rules to keep your code consistent and 
easy to read.  

Based on the last example we could accept the following formatting rules:

\begin{enumerate}
    \item Leave a space on both sides of addition operators.
    \item Leave a space on both sides of equal signs.
    \item Start and end each inline math environment with a space.
    \item No spaces between juxtaposed factors.
\end{enumerate}

These formatting rules are to make your code easier to parse and not required
to compile your document in to pdf.  It is just so much easier to read 
\verb!\( ax^2 + bx + c = 0 \)! than \verb!\(ax^2+bx+c=0\)!.

